\chapter{Ausblick}
Die Studie zeigt uns, dass sich Lernende in Gruppen anhand von ihren Blickbewegungen in einem gewissen Maße einteilen lassen. Diese Einteilung lässt aber keine direkten Schlüsse auf die mathematische Kompetenz der Probanden zu.  Im Bereich der Verwendung von unterschiedlichen Darstellungsformen, konnten positive Auswirkungen, von essenziellen Bildtypen gezeigt werden.  Aus diesen Ergebnissen sollten sich Lehrkräfte immer bewusst sein, welche Unterstützung sie bei einer Aufgabe geben Möchten. Dies bietet den Lehrkräften eine weitere Differenzierungsmöglichkeit zu der in der Einleitung angeschnittenen Prozessdimension (cognitive process Dimension) und der Wissensdimension (Knowledge Dimension), welche von Anderson und Krakwohl aufgestellt wurde. Ebenso ist es gut möglich die SuS selbst im Laufe eines Arbeitsauftrages zu einer solchen Graphik kommen zu lassen, beispielsweiße mit einer Skitze oder einem Graphen. Aufgabentypen wie gerade beschrieben schaffen eine Anschauliche Brücke zwischen der Modelierung eines Problems bis hin zu der Lösung dieses Problems.


Mich hat die Auswertung der Studie vorerst überrascht, da ich nach der Einteilung der Gruppen vorerst davon ausgegangen bin, das die SuS, welche sich auf die Abbildungen konzentrieren und somit in die Gruppierung Problemlöser gehören, besser sind. In meiner Unterrichtserfahrung habe ich festgestellt, dass SuS am besten einen Zugang zu einem mathematischen Problem erhalten, wenn sie eine Anschauung zu dem Thema haben. Wie im Ergebnissteil bereits geschildert, ist meine Erfahrung nicht unbedingt falsch sondern kann es sehr gut an der Gruppeneinteilung liegen. 


Folgende Studien können diese Daten als Grundlage verwenden um sich neue Vorschungsfragen zu überlegen und hierbei Beispielsweise andere Unterteilungen verwenden. Eine Möglichkeit hierbei wäre natürlich die Auswertung in umgekehrter Reihnfolge anzugehen: Hierbei werden die Probanden mit guten Leistungen in eine Gruppe zusammengefasst und danach geschaut, ob sie Gemeinsamkeiten in ihren Eyetrackingdaten aufweißen. Somit würde die Gruppeneinteilung nicht aus vorüberlegten Ansätzen passieren sondern als Resultat der Auswertung. 
Ebenso wäre es Inerresant die gegebene Studie mit einer deutlich größeren Anzahl an Probanden durchzuführen, um die Statistische Aussagekraft der Studie zu verbessern, da in vorliegender Studie lediglich die Gruppen aus 6-10 Probanden bestanden hat.