\chapter{Diskussion}

Wie man aus der Graphik der Gruppenvarianz entnehmen kann unterscheiden sich die Gruppen untereinger in den Merkmalen: Fixationen auf den Text und das Bild jeweils nach dem ersten Durchgang und in der Dauer des zweiten Durchgangs. In der Dauer des ersten Durchgangs ist kein signifikanter Unterschied zwischen den Gruppen festzustellen ($\alpha$ = 0,620 F = 0,490). Dieses Ergebniss Stimmt mit der Theorie überein, dass erstmal der Text von den Probanden linear gelesen wird und erst danach sich die Unterschiede aufzeigen. In der Anzahl der Fixationen auf den Text sowie das Bild dem ersten Durchgang ist ein signifikanter Unterschied festzustellen ($\alpha$ = 0,000 F = 11,991 für das Bild und $\alpha$ = 0,001 F = 10,034 für den Text). Ebenso ist in der Dauer nach dem ersten Durchgang ein Signifikanter Unterschied zu erkennen ($\alpha$ = 0,000 F = 17,073). Also ist die Forschungsfrage (1) mit ja, man kann die Probanden in Gruppen unterscheiden, zu beantworten.

Verwunderlich ist bei der Auswertung der Mittelwerte der Punkte (Forschungsfrage 2), dass die Problemlöser mit 5,5 Punkten im Schnitt am schlechtesten unter den Gruppen abgeschlossen haben. Dieses Ergebniss ist dahingehend verwunderlich, da in der Lerntypeinteilung von Gangé die Problemlöser, welche sich viel mit den unterschiedlichen Darstellungsformen auseinander setzen, hierachisch über den Textuellen, welche sich aus den Grundformen des Lernes nach Gané zusammen setzen sollten. Auf der anderen Seite hat Zech seine Lerntypeinteilung nicht mehr hierarchisch eingeteilt, was die Ergebnisse weniger verwunderlich erscheinen lässt. Typ Textuell hat in dieser Auswertung mit 6,3 Punkten im Mittel am besten besten Abgeschlossen und er Typ Problemlöser liegt mit 6,17 Punkten im Mittelfeld. Die Punktevarianzabbildung zeigt aber auch, dass die Unterschiede bei den Punkten der einzelnen Gruppen nicht Signifikant sind ($\alpha$ = 0,551 F = 0,615). 

Der folgende Ausdurck von Hyrskykari, Ovaska, Majaranta, Räihä und Lehtinen (2008) erklärt anschaulich was wie leicht es ist Fehlbeurteilungen bei der Untersuchung von Eyetrackingdaten anzustellen:

``Eine markante Zone auf einer Heatmap wird oft als eine interressante Zone interpretiert. Es hat die Aufmerksamkeit des Benutzers auf sich gezogen und deswegen wird angenommen, dass die Zone für den Benutzer jetzt verständlich ist. Dennoch kann aber auch das Gegenteil der Fall sein: Die Zone hat die Aufmerksamkeit des Benutzers angeregt, weil sie verwirrend und problematisch ist und der Benutzer die dargestellte Information nicht(fett) verstanden hat''. (%\cite)

Ebenso haben Similarly, Ballard, Hayhoe und Sullivan (2003) in ihrer Studie gezeigt, dass in manchen Situationen Benutzer direkt auf ein relevantes und wichtiges Objekt schauen und dennoch keine Gedächtnisspur (Wissensgewinn) regestrierbar ist. 

Daraus lässt sich schleißen, dass es nicht so einfach ist die Verhaltensmuster (in unserer Studie Blichbewegungen) der Probanden direkt in ihre kognitiven Prozesse zu übersetzen. Somit ist es leicht sich von den Eyetrackingdaten blenden zu lassen und diese falsch zu interpretieren. In dieser Studie könnten die Probanden aus der Gruppe Problemlöser lediglich viele Fixationen auf den Bildern haben weil sie diese nicht ganz verstanden haben. 

Ebenso ist die Gruppengröße von 6 Probanden in Gruppe Unsicher, 10 Probanden in Gruppe Problemlöser und 6 Probanden in Gruppe Textuell deutlich zu klein um signifikante Ergebnisse zu erzeugen. Man könnte auch sagen diese Gruppengröße hat zu wenig Statistische Aussagekraft um eindeutige Aussagen zu treffen. 

Wenn man sich der dritten Forschungsfrage widmet: In wie fern die Bilder einen Einfluss auf das Ergebniss der Probanden hat. Stellt man schon in den ersten beiden Aufgaben einen Wiederspruch fest: Radfahrerin 1 und Radfahrerin 2 verwenden beide ein dekoratives Bild und ihre Beantwortung fällt im ersten Aufgabenteil sehr gut mit 97,5 Prozent aus und im zweiten Teil, bei dem das selbe dekorative Bild verwendet wurde sehr schlecht mit 17,9 Prozent aus. Hierbei ist durch den Unterschied in der Beantwortung sehr schnell feststellbar, das sich das Niveau der Aufgabenstellung deutlich von dem ersten zu dem zweiten Aufgabenteil unterscheiden muss.

Die beiden Bergaufgaben wurden zu 76,9 und zu 74,3 Prozent richtig bearbeitet, obwohl das beigelegte Bild ebenso nur dekorativ ist. In restlichen Fällen war die verwendete Graphik jeweils essenziell und wurden mit Autofahren 94,9 Prozent, Zufall 51,3 Prozent und Rennfahrer 76,9 Prozent richtig beantwortet. Hierbei wird der Aufgabenteil Zufall auffällig, da er am schlechtesten von den Aufgabenstellungen mit essenziellen Graphiken bewertet wurde. 

Im Durchschnitt über alle Probanden wurden die Aufgaben mit dekorations Bildern zu 66,6 Prozent richtig beantwortet und die Aufgabe mit essenziellen Bildern zu 74,4 Prozent. Dies ist zwar kein signifikanter Untersied ($\alpha$ = 0,749 F = 0,115), zeigt aber eine Tendenz von dem  Effekt, aus der Studie von Matthias Böckmann und Stanislaw Schukajlow (2018). Mögliche Ursachen wieso der Effekt nicht so groß ist wie erwartet sind: Zum einen wie bereits im ersten Teil geschildert ist der Schwierigkeitsgrad der einzelnen Aufgaben sehr unterschiedlich. Zum anderen ist es ein Unterschied, ob man aus einer Graphik lediglich Werte ablesen muss, wie es in Aufgabe Autofahren der Fall ist, oder die Werte stochastisch auszuwerten, wie es in der Aufgabe Zufall der Fall ist. 

Bei der letzten Forschungsfrage (4) wird gefragt ob bestimmte Aufgabentypen für einzelne Probandengruppen positive oder auch negative Auwirkungen haben. Wie wir in dem Ergebnissteil gesehen haben, werden hierbei drei Fälle einer weiteren Betrachtung unterzogen und versucht eine Erklärung hierfür abzugeben. 

Beginnend mit der schlechten Bearbeitung der Problemlöser, bei der Aufgabe Autofahrt ($\alpha$ = 0,243 F = 1,407).  Bei dieser Aufgabenstellung war lediglich eine eine Graphik abgebildert, der die Geschwindlichkein mit der Zeit der Autofahrerin in Relation setzt. Gefragt war, ob der linke und höhere Teil (also mehr Geschwindlichkeit)  des Graphen eine größere Strecke darstellt, wie der gleichbeite (also gleiche Zeitdauer) rechte Teil des Graphen.  Hierbei bedeutet der also der der niedrigere Teil (rechts) eine kürzere zurückgelegte Strecke, da  (Weg) x = (Geschwindlichkeit)  v mal t (Zeit) gilt. In der Einstiegsfolie, nach der die Gruppen eingeteilt wurden, war jeweils in den Graphiken, welche sie viel betachtet haben, ganz Eindeutig zu sehen, was mit ihnen gemeint ist. In der Graphik dieser Aufgabe, kann man die gleichbreiten (Zeit) Graphen auch als gleich Weit aus der des zurückgelegten Weges missverstehen. Dieser Effekt stärker jedoch nicht signifikant stark auf, wenn Graphiken problematisch auf einen wirken, wie es in der Gruppierung Problemlöser der Fall ist.

Im zweiten Fall wurde Zufall von dem Typ Unsicher besonders gut gelöst($\alpha$ = 0,152 F = 2,135). In dem Aufgabenteil Zufall sind zwei Graphiken sehen, welche beide ausgewertet und verknüpft werden müssen um die Aufgabenstellung richtig zu beantworten. Der Typ Unsicher hat sich ausgezeichnet indem er auf ersten Folie eine besonders lange zweite Runde mit vielen Fixationen auf Bild und Text gezeigt hat. Diese Dauer kann auch als Gewissenhaft einstuft werden, was ihm in der Aufgabe Zufall zum Vorteil wird, da die beiden Graphiken zuerst abgezählt und dann verküpft werden müssen bevor die Aufgabe beantwortet werden kann. Hierbei kann somit der Unsichere mit einer längeren Bearbeitungszeit ein nicht signifikantes aber durchschnittlich besseres Ergebnis erziehlen. 

Im letzten Teil wurde der Aufgabentyp Rennfahrer von der Gruppe Textuell am besten gelöst ($\alpha$ = 0,092 F = 2,990). Die Aufgabenstellung war: aus einem gegebenen Graph, welcher die Geschwindlichkeit und die Steckenentfernung in Relation gesetzt hat, den Abstand bis zur längsten geraden Strecke anzugeben. Hierbei konnte dieser Wert einfach aus dem Graphen abgelesen werden, da dieser Wert am Anfang des längsten Teiles ohne Kurve, also Geschwindlichkeits verlust, liegt. Die Gruppe Textuell hat sich auf der ersten Folie dadurch auszezeichnet, dass sie viele Fixationen auf den Text hat. Da der Abstraktionsgrad der Aufgabe Zufall nicht besoders hoch ist, kann durch gewissenhaftest Lesen der Aufgabestellung (des Textes) und einfaches Ablesen des Graphen diese Aufgabe erfolgsversprechend gelöst werden. 

Abschließend wenden wir uns noch dem Ergebniss aus der Tabelle XY zu, welche die verschiedenen Lerntypen mit den Bildtypen in Relation setzt. Hierbei fällt auf, das die Unterschiede unter die Gruppen sehr gering sind im teil der dekorativen Bilder ($\alpha$ = 0,897 F = 0,109), jedoch im Teil der essenziellen Bilder der Typ Problemlöser schlechter abgeschnitten hat ($\alpha$ = 0,130 F = 2,287). Dieses Ergebniss steht im Einklang mit dem der Vermutung aus der Disskusion aus Forschungsfrage 1, welche besagt, dass die Problemlöser nicht viele Fixationen auf den Bildern hatten, weil sie sie Verstanden haben, sondern weil es ihnen schwergefallen ist mit ihnen um zu gehen. Auf den Selben Schluss kann man auch bei der Tabelle kommen, das der Typ Problemlöser nicht sehr gut mit den Graphiken arbeiten kann und deswengen essenzielle Bilder schlechter bearbeitet, als die beiden anderen Lerntypen. Diese Ergebnisse sind leider wiederrum leider nicht signifikant und geben somit lediglich eine Tendenz an. 
